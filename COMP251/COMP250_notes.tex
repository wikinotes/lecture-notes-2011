\documentclass[9pt, letterpaper, oneside]{article}

\usepackage{fancyhdr}
\setlength{\headheight}{15.2pt}
\setlength{\headwidth}{500pt}
\pagestyle{fancyplain}

\usepackage[parfill]{parskip} 
\usepackage{graphicx}
\usepackage{amsmath}
\usepackage{amssymb}
\usepackage{epstopdf}
\usepackage{fullpage}
\usepackage[linktoc=section, colorlinks=true, urlcolor=blue]{hyperref}
%\setlength{\parindent}{0pt}
%\setlength{\parskip}{1ex plus 0.5ex minus 0.2ex}
\setcounter{secnumdepth}{3}
\setcounter{tocdepth}{3}


\DeclareGraphicsRule{.tif}{png}{.png}{`convert #1 `dirname #1`/`basename #1 .tif`.png}
\date{}
\begin{document}
\lhead{Brought to you by WikiNotes. Join our \href{http://www.facebook.com/home.php?sk=group_166578420027385&ap=1}{facebook group} or take a look at our website on \href{http://www.wikinotes.ca}{wikinotes.ca}\newline}

\section*{}
\begin{center}
\section*{COMP 251 by Andrea Sigler}
\small{McGill University - Fall 2011} 
\end{center}

%%% Thursday September 1st 2011

\paragraph{Course Information}

\begin{itemize}
\item When/Where: Tuesdays and Thursdays 2:35-4:00, Trottier 0100
\item Instructor: Andrea Sigler, adrea.sigler@mail.mcgill.ca
\item Textbook: Introduction to Algorithms, Cormen Leisterson
\item Material: Graphs, Greedy, Divide+Conquer, Dynamic Programming, Network Flow
\item Prerequisites: COMP 250
\item Grading:
	\begin{itemize}
		\item 4 $\times$ 10\% homework
		\item 1 $\times$ 10\% midterm
		\item 50\% final
	\end{itemize}
\end{itemize}

\subparagraph{Introduction to the Course}
\newline
\begin{itemize}
	\item Lists
		\begin{itemize}
			\item Insert
			\item Remove
		\end{itemize}
	\item Arrays
		\begin{itemize}
			\item Access in O(1)
		\end{itemize}
	\item Sets
		\begin{itemize}
			\item Intersect
			\item Union
		\end{itemize}
	\item Algorithm
		\begin{itemize}
			\item Finite set of operations given many goals
			\item Complexity: The amount of time and memory in correlation with the input required.
		\end{itemize}
	\item Turing machine
		\begin{itemize}
			\item Made by Alan Turing in 1936
			\item
			\item Can read, write, move left, move right.
		\end{itemize}
	\item Time Complexity
		\begin{itemize}
			\item uniform cost model
			\item bit model
			\item uses of an object
				\begin{itemize}
					\item given n coins and 1 balance
					\item you are guaranteed all coins are the same weight except one is heavier
					\item suggested solution: divide the pile in two and take the heaviest one, divide again until you reach one coin - complexity is $\log n$. In this case worst case scenario = best case scenario.
					\item other suggested solution: put two coins on each side, then keep on adding one on each side until there's a heavier side. Best-case scenario = 1, worst case scenario = $n / 2 \forsome $
				\end{itemize}
		\end{itemize}
\end{itemize}

\end{document}
