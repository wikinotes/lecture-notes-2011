\documentclass[9pt, letterpaper, oneside]{article}

\usepackage{fancyhdr}
\setlength{\headheight}{15.2pt}
\setlength{\headwidth}{500pt}
\pagestyle{fancyplain}

\usepackage[parfill]{parskip} 
\usepackage{graphicx}
\usepackage{amsmath}
\usepackage{amssymb}
\usepackage{epstopdf}
\usepackage{fullpage}
\usepackage[linktoc=section, colorlinks=true, linkcolor=blue, urlcolor=blue]{hyperref}
%\setlength{\parindent}{0pt}
%\setlength{\parskip}{1ex plus 0.5ex minus 0.2ex}
\setcounter{secnumdepth}{3}
\setcounter{tocdepth}{3}
\newcommand*\dbl{\leftrightarrow}


\title{MATH 240 - Discrete Structures}
\author{McGill University \\ Fall 2011}
\date{}
\DeclareGraphicsRule{.tif}{png}{.png}{`convert #1 `dirname #1`/`basename #1 .tif`.png}

\begin{document}
\lhead{Brought to you by WikiNotes. Join our \href{http://www.facebook.com/home.php?sk=group_166578420027385&ap=1}{facebook group} or take a look at our website on \href{http://www.wikinotes.ca}{wikinotes.ca}\newline}

\maketitle
\tableofcontents
\addcontentsline{toc}{section}{Course Information}

%%% Friday September 2nd 2011

\section*{Course Information}

\begin{itemize}
\item When/Where: MWF 10:35-11:35, Stewart Bio N2/2
\item Instructor: Sergey Norin math.mcgill.ca/~snorin
\item Textbook: Discrete Mathematics, Elementary and Beyond by Lovasz, Pelikan and Vesztergombi
\item Prerequisites: 
\item Grading:
	\begin{itemize}
		\item 20 \% assignments 20 \% midterm and 60 \% final
		\item 20 \% assignments 80 \% final
		\item (best of two above)
	\end{itemize}
\end{itemize}

\section*{Introduction}
Discrete vs. Continuous structures
\begin{itemize}
	\item Objects in discrete structures are individual and separable
	\item An intuitive analogy is that discrete structures focus on individual trees in the forest whereas continuous structures care about the landscape airplane view.
	\item Discrete structure courses can be called "computer science semantics" in other universities. Mathematics for computer science.
	\item Naive examples
	\begin{itemize}
		\item Counting techniques: There are two ice cream shops. One sells 20 different flavours whereas the other offers 1000 different combinations of three flavours. Which one has the most possible combinations of three flavours?
		\item Cryptography: Two parties want to communicate securely over an insecure channel. Can they do it? Yes, using number theory. Discrete Structures are used in cryptography (what this question is about), coding theorem (compression of data) and optimization.
		\item Graph Theory: Suppose you have 6 cities and you want to connect them with roads joining the least possible number of pairs, so that every pair is connected, perhaps indirectly. In how many ways can we connect these cities using 5 roads?
	\end{itemize}
	\item Before we address these problems, we must agree upon a language to formalize them.
\end{itemize}

\section{Sets}

\subsection{Definition}
A set is a collection of distinct objects which are called the elements of the set.

Examples: We use a capital letter for sets.
\begin{itemize}
	\item $A = \{ Alice, Bob, Claire, Eve \}$
	\item $B = \{ a, e, i, o, u \} = \{ o, i, e, a, u\}$
	\item $\mathbb{N} = \{1, 2, 3, 4, 5, ...\}$ (natural numbers)
	\item $\mathbb{Z} = \{.., -2, -1, 0, 1, 2, ..\}$ (integers)
	\item $\emptyset = \{\} $ (no elements, note: $\{\emptyset \} \neq \emptyset \}$)
	\item If x is an element of A we write $x \in A$ which is read "belongs", "is an element of" or "is in" e.g. $Alice \in A, Alice \notin \mathbb{N}$
	\item We say that X is a subset of a set Y if for every $z \in X$ we have $z \in Y$ Notation: $X \subseteq Y$.
	\item $\emptyset \subseteq \{1,2,3,4,5\} \subseteq \mathbb{N} \subseteq \mathbb{Z} \subseteq \mathbb{Q} \subseteq \mathbb{R} $
\end{itemize}

\subsection{Operations on sets}

$U = \{1,2,3,4,5,6 .. 10\} = \{ x \in \mathbb{N}: x \leq 10\}$

$A = \{2,4,6,8,10\} = \{x \in U: x $ is even$ \}$

$B = \{2,3,5,7\} = \{x \in U: x $ is prime$ \}$

An intersection $A \cap B$ is a set of all elements belonging to both A or B: $A \cap B = \{2\}$

A union $A \cup B$ is a set of all elements belonging to either A or B: $A \cap B = \{2,3,4,5,6,7,8,10\}$

$|A| = 5, |B| = 4, |A \cap B| = 1, |A \cup B| = 8 |\emptyset| = 0, |\mathbb{N}| = \infty$

$A - B$: all elements of A which do not belong to B $\{x : x \in A, x \notin B\}$

$A \oplus B, A \triangle B$: symmetric difference, set of all elements belonging to exactly one of A and B

\subsection{Venn Diagrams}

A way of depicting all possible relations between a collection of sets. For a set A, $|A|$ denotes the number of elements in it. 

Typically, Venn diagrams are useful for 2 or 3 sets.

\subsection{Theorems}

\begin{itemize}
	\item $A \cap (B \cup C) = (A \cap B) \cup (A \cap C)$
	\begin{itemize}
		\item Fact: For any two finites sets $|A| + |B| = |A \cap B| + |A \cup B|$
		\item Proof:
		\begin{enumerate}
			\item $x \in A \cap (B \cup C)$ then $x \in (A \cap B) \cup (A \cap C)$
			\begin{itemize}
				\item $x \in A$ and $(x \in B$ or $x \in C)$
				\item if $x \in B$ then $x \in (A \cap B)$ therefore $x \in (A \cap B) \cup (A \cap C)$
				\item if $x \in C$ then $x \in (A \cap C)$ therefore $x \in (A \cap B) \cup (A \cap C)$
			\end{itemize}
			\item $x \in (A \cap B) \cup (A \cap C)$ then $x \in A \cap (B \cup C)$
				\begin{itemize}
					\item $x \in (A \cap B)$ therefore $x \in A$ and $x \in (B \cup C)$
				\end{itemize}
		\end{enumerate}

	\end{itemize}
		\item $A \oplus B = (A \cup B)-(A \cap B) = (A - B) \cup (B - A)$
\end{itemize}

% Wednesday September 7th


\section{Logic}

Way of formally organizing knowledge studies inference rules i.e. which arguments are valid and which are fallacies.

\subsection{Propositional Calculus}

A proposition is a statement (sentence) which is either true or false.

Some examples:
\begin{itemize}
	\item $2 + 2 = 4 \to$ true
	\item $2+3 = 7 \to$ false
	\item "If it is sunny tomorrow, I will go to the beach." $\to$ valid proposition
	\item "What is going on?" $\to$ not a proposition
	\item "Stop at the red light" $\to$ not a proposition
	\item We are given 4 cards. Each card has a letter (A-Z) on one side, a number (0-9) on the other side. "If a card has a vowel on one side then it has an even number on the other" Two ways to refute this proposition: Either turn over a vowel card and find an odd number. Or turn over an odd number and find a vowel.
\end{itemize}

\subsection{Notation}
\begin{itemize}
	\item Letters will be used to denote statements: p, q, r
	\item $p \wedge q$: "and", "conjunction", "p and q" (are both true)
	\item $p \vee q$: "or", "disjunction", "either p or q" (is true)
	\item $\neg p$: "not", "p is false"	
\end{itemize}

\subsection{Truth Tables}

\subsection{Rules of Logic}

\begin{enumerate}
	\item Double negation: $\neg (\neg p) \leftrightarrow p$
	\item Indempotent rules: $p \wedge p \leftrightarrow p \qquad p \vee p \leftrightarrow p$
	\item Absorption rules: $p \wedge (p \vee q) \leftrightarrow p \qquad p \vee (p \wedge q) \leftrightarrow p$
	\item Commutative rules: $p \wedge q \leftrightarrow q \wedge p \qquad p \vee q \leftrightarrow q \vee p $
	\item Associative rules: $p \wedge (q \wedge r) \leftrightarrow (q \wedge p) \wedge r \qquad p \vee (q \vee r)\leftrightarrow (p \vee q) \vee r$
%% Friday Sept 9th 2011
	\item Distributive rules: $p \wedge (q \vee r) \leftrightarrow (p \wedge q) \vee (p \wedge r) \qquad p \vee (q \wedge r) \leftrightarrow (p \vee q) \wedge (p \vee r)$
	\item De Morgan's rule: $\neg((\neg p) \vee (\neg q)) \leftrightarrow p \wedge q \qquad \neg((\neg p) \wedge (\neg q)) \leftrightarrow p \vee q$\\
	$p \vee (\neg ((\neg p) \wedge (\neg q))) \leftrightarrow  p \vee (p \vee q) \leftrightarrow (p \vee p) \vee q \leftrightarrow p \vee q$
\end{enumerate}


\subsubsection{Conditional Statements}

\begin{enumerate}
	\item $p \to q$
	\begin{itemize}
	\item Theorem: if (an assumption holds), then (the conclusion holds).
	\item Implication: "if p then q"\\
		p = "a, b, \& c are two sides and the hypthenuse of a triangle"\\
		q = "$a^2 + b^2 = c^2$"
	\item $p \to q$ "If p then q" p implies q, p is sufficient for q\\
		$(p \to q) \leftrightarrow (q \vee (\neg p))$
		% insert truth table 1
	\item Examples: 
		\begin{itemize}
			\item "If the Riemann hypothesis is true then 2 + 2 = 4" TRUE\\
			p = "the Riemann hypothesis"\\
			q = "2+2=4"\\
			True proposition is implied by any proposition.
			\item "If pigs can fly then pigs can get sun burned" TRUE\\
			False statement implies any statement
			\item "If 2+2 =4 then pigs can fly" FALSE\\
			The implication is false only if the assumption holds and the conclusion does not.
		\end{itemize}
		% insert truth table 2
	\item $p \to q \leftrightarrow (\neg p) \to (\neg q)$
	\item $(p \to q) \wedge (q \to p) \leftrightarrow (p \leftrightarrow q)$
	\end{itemize}
\end{enumerate}

\paragraph{Puzzle}

There are three boxes A, B, C. Exactly one contains gold in it.
\begin{itemize}
	\item Box A: Gold is not in this box
	\item Box B: Gold is no in this box
	\item Box C: Gold is in box A
\end{itemize}
Exactly one of these propositions is true. Where is the gold?
Let us formalize the propositions.
\begin{itemize}
	\item p:  "Gold is in box A"
	\item q:  "Gold is in box B"
	\item r:  "Gold is in box C"
	\item Box A: $q \vee r$
	\item Box B: $p \vee r$
	\item Box C: p
	\item $p \to (p \vee r)$
	\item $\neg (p \vee r) \to q$
\end{itemize}

%%% Monday Sept 12th

\subsection{Tautologies \& Contradictions}

\paragraph{Definition}
\begin{itemize}
	\item A \textbf{tautology} is a statement that is always true (the rightmost column of the corresponding truth table has T in every row) e.g. $p \vee (\neg p)$
	\item A \textbf{contradiction} is a statement that is always false e.g. $p \wedge (\neg p)$
\end{itemize}

\paragraph{Notation}
\begin{itemize}
	\item 1 denotes a tautology
	\item 0 denotes a contradiction
	\item $1 \vee p \leftrightarrow 1$
	\item $0 \vee p \leftrightarrow p$
	\item $1 \wedge p \leftrightarrow p$
	\item $0 \wedge p \leftrightarrow 0$
\end{itemize}

\begin{itemize}
\item $p \wedge (p \vee q)$\\
\begin{tabular}{| l | l | l | l | }
  p & 1 & $p \vee q$  & $p \wedge (p \vee q)$\\
  T & T & T & T \\
  T & F & T & T \\
  F & T & T & F \\
  F & F & F & F \\
\end{tabular} \\
$\to$ Not a tautology and not a contradiction \\
$p \wedge (p \vee q) \leftrightarrow p$ (one of the rules)
\item $p \vee (p \wedge q) \vee (p \to q) \leftrightarrow (p \vee (p \wedge q)) \vee (p \to q)$ \\
$(p \to q) \leftrightarrow (\neg p) \vee q \leftrightarrow p \vee (p \to q)$\\
$\leftrightarrow p \vee ((\neg p) \vee q) (absorption)$\\
$\leftrightarrow (p \vee (\neg p))\vee q$\\
$\leftrightarrow 1 \vee q \leftrightarrow 1$
\end{itemize}

\subsection{Proofs}
\begin{itemize}
	\item $(p \to q) \wedge (q \to r) \to (p \to r)$ (always true)
	\item Implication is transitive: $p \to q \to r$
	\item A \textbf{proof} of a conclusion q given premise p is a sequence of implications (valid) $p \to p_2 \to p_3 \to .. \to p_k \to q$
	\item To prove $(p \leftrightarrow q)$ \\
		$\qquad (p \leftrightarrow q) \leftrightarrow (p \to q) \wedge (q \to p)$
	\item Theorem: Let p(x) be a polynomial then p(0) = 0 if and only if p(x) = x q(x) for some polynomial q(x)
	\item Proof: "p(0) = 0" and "p(x) = x q(x) for some polynomial q(x)"
	\begin{enumerate}
		\item $p(x) = a_nx^n + a_{n -1}x^{n-1} + ... + a_1x + a_0$ \\
			$p(0) = 0 \to a_0 = 0 \to$ \\
			$p(x) = a_nx^n + a_{n -1}x^{n-1} + ... + a_1x \to$ \\
			$p(x) = x(a_nx^{n-1} + a_{n -1}x^{n-2} + ... + a_1$ \\
			$p(x) = xq(x)$ \\
			$q(x) = a)nx^{n-1} + a_{n-1}x^{n-2} + ... + a_2$ \\
			True so proven.
		\item $p(x) = x q(x) \to p(0) = 0 \cdot q(0) \to q(0) = 0$
	\end{enumerate}
	\item Proof by contradiction: $(p \to q) \leftrightarrow ((\neg q)\to (\neg p))$
	\item Pigeonhole principle: We place an objects into m bins. If $n > m$ then some bin contains at least 2 objects.
	\item Proof: p = "$n > m$" and q = "Some bin contains at least 2 objects" \\
	$\neg q$ = "every bin contains at most 1 object" \\
	$\neg p$ = "$n \leq m$"
	$\neg q \to\ \neg p$ is trivial
	\item Theorem: There are infinitely many prime numbers \\
		Direct proof of this theorem is unlikely, there is no known simple formula producing prime numbers
	\item Proof: Assume $\neg p$. There are infinitely many prime numbers $p_1, p_2, p_3 .. p_k$ \\
	Consider $p = p_1p_2...p_k + 1$
	Every integer greater than 1 is divisible by a prime. (Prime number is the integer divisible by only 1 and itself).
	Suppose $p = p_im$ for some $1 \leq i \leq k$ and an integer m, then $p_i(p_1p_2...p_{i-1}p_{i+1}...p_k) + 1 = p_im$
	$p_i(m - p_1p_2..p_k) = 1$ (except $p_1$)
	"1 is divisible by $p_i$, a contradiction"
\end{itemize}

%%% Wednesday September 14th

\section{Circuit Complexity}

\subsection{Boolean Logic}

\begin{itemize}
	\item \textbf{Objects}: statements p, 1
	\item \textbf{Operators}: $\vee, \wedge, \neg$, etc
\end{itemize}

\subsection{Logic Gates}

Will insert logic gate diagrams later when I figure how to insert images.
%%% INSERT LOGIC GATES DIAGRAMS

\begin{tabular}{| l | l | l | l | l | }
  p & q & r  & $p \oplus q$ & $(p \oplus q) \oplus r$\\
  1 & 1 & 1 & 0 & 1 \\
  1 & 1 & 0 & 0 & 0\\
  1 & 0 & 1 & 1 & 0\\
  1 & 0 & 0 & 1 & 1\\
  0 & 1 & 1 & 0 & 0 \\
  0 & 1 & 0 & 1 & 1\\
  0 & 0 & 1 & 1 & 1\\
  0 & 0 & 0 & 0 & 0\\
\end{tabular} \\

\paragraph{Majority Circuit (for 3 inputs)}

p, q, r $\to$ 
$\begin{cases}
	$1 (or T) if at least 2 of p, q \& r are 1's$ \\
	$0 (or F) otherwise$
\end{cases}$

\paragraph{Size}
A logical circuit has size equal to the number of gates in it and depth equal to the length (or number of gates) of the longest path from an input to the final output.

Given a boolean formula, what is the minimum size (or depth) of a circuit necessary to compute it?
(depth is frequently assumed to be constant).

Given a circuit C with inputs $p_1, p_2, ..., p_n$

Can we test if C is always a contradiction? The answer is trivially yes, if we test all possible inputs. It would take $2^n$.

\subsection{Algorithms}
\begin{itemize}
	\item Every logic formula can be represented as a combinational circuit
	\item Can we represent a given formula by a "simple" circuit
	\item Given a circuit (with inputs $p_1, p_2, ..., p_n$ can we test quickly if C is a contradiction? (we can test in $2^n steps$
	\item \textbf{Algorithm}: A step-by-step procedure for solving a problem, precise enough to be carried out on a computer
\end{itemize}

%%% MISSED

%%% Friday September 16th
\section{Polytime algorithms P\# NP conjecture}

\subsection{Definition}
Given algorithm A its running time $t_A(n) =$ maximum number of steps the algorithm can require on inputs of size n

A is a \textbf{polynomial time} algorithm if $t_A(n)$ is polynomially bounded ($t_A(n)=O(n^2)) \dbl$ fast, efficient

P is class of problem which allow polynomial time algorithms.

\paragraph{Examples}

\begin{enumerate}
	\item Evaluating the median of a set of numbers
	\begin{itemize}
		\item Problem: $x_1, x_2, .., x_n \leftarrow$ Input
		\item Question: decide whether the median of the list is $\leq 1000$
		\item Algorithm: 
		\begin{itemize}
			\item Sort the list going once through the list ($\leq n$ steps) we can find smallest $x_i$
			\item Repeat to find the second smallest number and so on
			\item Requires $O(n^2)$ time to sort
			\item Check if $x_{\frac{n}{2}} ( x_{\frac{n}{2}})$ is at most 1000 (roughly $n^2$ steps polytime).
		\end{itemize}
	\end{itemize}
	\item Multiplication
		\begin{itemize}
			\item Input: $2 n$ digit numbers
			\item Output: $a \times b$
			\\ roughly $n^2$ steps
		\end{itemize}
	\item Problem Factoring
		\begin{itemize}
			\item Input: a composite number C
			\item Output: Find natural numbers $a, b > 1, c = a \times b$
			\item Brute-Force search: Try all prime numbers up to C. Time: $10^{n/2} \to$ exponential time algorithm
			\item RSA ran contests until 2007 offering prizes for factoring (roughly 20 computer years for factoring 200 digit numbers)
		\end{itemize}
\end{enumerate}

\subsection{NP problems (non-deterministic polynomial time)}

\begin{itemize}
\item
A \textbf{decision problem} is a problem with a yes/no answer.
Example: 
	\begin{itemize}
	\item Input: a combinatorial circuit (with n inputs)
	\item Output: Is C \textbf{not} a contradiction?
	\end{itemize}
\item
A decision problem is in the class NP if a "yes" answer always has a certificate which can be verified in polynomial time.
\item
A problem is in NP when the answer is positive. A magician can quickly convince you that it is e.g. "testing that a circuit is not a contradiction" is in NP.
\item
If there exists a set of values for inputs so that the circuit outputs 1 (or T) then given this collection of inputs verifying that it works is fast.
\end{itemize}

\paragraph{Examples}
\begin{enumerate}
\item Factoring:
	\begin{itemize}
	\item Input: n digit number
	\item output: Is this number composite and if it is, factor it.
	\end{itemize}
\item Traveling Salesman problem:
	\begin{itemize}
	\item Input: Collection of n cities and distances between them
	\item Travelling salesman tour: An ordering of cities $c_1 \to c_2 \to ... \to c_n$ visiting each city once
	\item Question: Is there a tour of total length $\leq 1000$ miles $\to$ is in NP
	\end{itemize}
\end{enumerate}

\subsection{$P \neq NP$}

There exist problems which cannot be solved efficiently but for which a positive answer can be verified efficiently. There exists problems for which brute-force search is essentially the best possible strategy. If there are problems where you need a magician, then it is NP.

If there exists a problem in NP but not in P (if the conjecture is true) then testing if a circuit is a contradiction, travelling salesman problem, and a very large class of similar problems are all not in P

If P = NP then airline scheduling, protein folding, packing boxes, finding short proof for theorems all can be done efficiently but certain cryptography becomes impossible.

The universal opinion is that $P \neq NP$

\subsection{Scott Aovonson's reasons for $P \neq NP$}

Empirical: Problems in NP remain heuristically hard, however problems which are now known to be in P (linear programming, primality testing) but efficient heuristics existed long before.

%%%%% Monday September 19th

\section{Proof Techniques: Predicate calculus}

\paragraph{Reminder}

A proof is a sequence of implications deriving a conclusion q from a premise p: $p \to q$

\begin{itemize}
\item Direct Proof: $p \to p_1 \to p_2 \to p_3 .\to ... \to p_k \to q$
\item Proof by contradiction: $p \to q \dbl (\neg q \to \neg p)$
\item Case Analysis: $(p \wedge q \to r) \dbl (p \to r) \wedge (q \to r)$ See below
\item Counter Examples: See below
\end{itemize}

\paragraph{Case Analysis}
\begin{itemize}
\item Proposition: For positive integer n: $3 \nmid n \to 3 \mid n^2 + 2$\\
	($a\mid b \to$ "a divides b" there exists an integer c, b = ac)
	Proof: Divide n by 3 with remainder
\end{itemize}

%% missed the proof

\paragraph{Couter Example}
	\begin{itemize}
		\item Proposition: $n^2 + n + 1$ is prime for every positive integer n $\leq$ 10
		\item $4^2 + 4 + 1 = 21 = 7 \cdot 3$
		\item This is a counter example: the statement is false
		\item Mathematical Notation
		\begin{itemize}
		\item $p \to q \wedge r \to p \to q$
		\item $\neg(p \to q) \to \neg(p \to q \wedge r)$ 
		\item q is a counter example to the implication "$n^2 + n + 1$ is prime for all integers n " 			\item "$n^2+ n + 1$ is prime" $\leftarrow$ P(n) predicate proposition depending on a variable
		$\forall n \in \mathbb{Z}(P(n))$\\
		Note: $\forall$  means "for all" e.g. "For all n in the set of integers the predicate "$n^2 + n + 1$"is prime" is true
		\item "There exists an integer n so that $n^2 + n + 1$ is not prime" is noted $\exists n \in \mathbb{Z}(Q(n))$ where Q(n) "$n^2 + n + 1$ is not prime" i.e. $Q(n) \neg P(n)$
		\end{itemize}
\end{itemize}

\paragraph{Goldback's conjecture}
Every even integer bigger than 2 is expressible as a sum of 2 primes.
\begin{itemize}
\item $\forall n \in$ "even integers", $n > 2 \to (\exists a, b \in \{primes\} (n = a + b))$)
\item
"71 is prime"
\item
$\forall a, b \in \mathbb{N} ( a \cdot b = 71) \to ((a = 1) \wedge (b=71))$
\end{itemize}

\paragraph{Limits}
\begin{itemize}
\item
"f(x) as a limit L as x $\to$ a" "$lim_{x \to a} f(x) = L$" As x approaches a f(x) becomes closer and closer to L"
\item
"For every $\epsilon > 0$, there exists $\delta > 0$ so that if $| x - a | < \delta then |f(x) - L |< \epsilon$"
\item
"$\forall \epsilon > 0 (\exists \delta > 0 (|x-a|<\delta \to |f(x) - L | < \epsilon))$
\item
"$lim_{x \to \infty} f(x) = L" \forall \epsilon > 0 (\exists X \cdot (\forall x > X (|f(x) - L| < \epsilon)))$
\end{itemize}

\paragraph{P(n) :  "$n^2 + n + 1$ is prime"}
\begin{itemize}
\item
$\neg(\forall n \in A : P(n)) \dbl \exists n \in A (\neg P(n))$
\item
$\forall n \in A : P(n) \dbl \neg(\exists n \in A (\neg P(n))$
\end{itemize}
\paragraph{"$\sin x$ does not have a limit as $x \to \infty$"}
\begin{align*}
\neg(\exists L : lim_{x \to \infty} \sin x = L)
&\dbl \forall L : (\neg (lim(\sin x) = L)\\
&\dbl \forall L (\neg(\forall \epsilon > 0 (\exists X (\forall x > X (|\sin x - L| <\epsilon)))))\\
&\dbl \forall L (\exists \epsilon > 0 (\neg (\exists X (\forall x > X (|\sin x - L| <\epsilon)))))\\
&\dbl \forall L (\exists \epsilon > 0 (\neg (\exists X (\forall x > X (|\sin x - L| <\epsilon)))))\\
&\dbl \forall L (\exists \epsilon > 0 (\forall X (\exists x > X(|\sin{x} -L | \geq \epsilon))))\\
\end{align*}

%%% Wednesday September 21st

\subsection{Divisibility Problem}
We want to prove the following theorem:
\begin{itemize}
\item Any collection of n+1 numbers chosen from the set \{1,2,...,2n\} contains two numbers so that one is divisible by the other.
\item $\forall n \in \mathbb{N} (\forall s \subseteq \{1,2,...,2n\} (|S| = n + 1) \to \exists a,b \in S ((a | b) \wedge (a + b)))$
\end{itemize}

\paragraph{Reminder: the pigeonhole principle} 

If $n+1$ objects are placed into n boxes then some box contains $\geq 2$ objects. To apply the principle we want to partition $\{1,2,...,2n\}$ into n subjects. 

\paragraph{Partition} We say that a collection $A_1, A_2, ... A_k$ of subsets of a set B is a \textbf{partition} of B if
\begin{enumerate}
	\item $\forall i,j : 1 \leq i < \leq k \qquad A_i \cap A_j = \emptyset$ (no element of B belongs to two different parts)
	\item $A_1 \cup A_2 \cup ... \cup A_k = B$
\end{enumerate}

Example: \{1,2,3,4,5,6,7,8\} \qquad \{1,2,4,6,8\} , \{3, 5\} , \{7\}

\paragraph{Proof} By the pigeonhole principle it suffices to find a partition $A_1, A_2, ... A_n$ of \{1,2,...,2n\} so that $(\forall i (\forall a,b \in A_i (a|b \vee b|a)))$

Here is a construction:
$A_i = \{(2i + 1), 2(2i -1), 4(2i -1), ..., 2^m(2i-1)\}$ up to maximum m: $2^m (2i-1) \leq 2n$
\begin{enumerate}
\item $A_i$ satisfies the desired property for all i
\item $A_1, A_2, .., A_n$ is a partition of \{1,2,...,2n\} \\
Ever positive integer can be uniquely written in a form $2^m(2i - 1)$ for some $i \geq 1, m \geq 0$
\end{enumerate}

Note: Is it true for some n:
"Every collection of n numbers chosen from \{1,2,...,2n\} contains 2 numbers one dividing the other"?

Counter-example: $n = 2 \quad \{1,2,3,4\} \to \{3,4\}$

%%% missed big chunk here


\subsection{Strangers and Clubs}

For a collection of people any two of them either have met or haven't . A club is a group of people who have pairwise met each other. A group of strangers is a group of people who pairwise have not met each other

Theorem: In any collection of 6 people rhwew is either a club of 3 people or a group of 3 strangers.

Proof Let x be one of the people in the collection. The following cases apply
\begin{enumerate}
	\item x has at least 3 acquaintances
	\begin{enumerate}
		\item Some two of acquaintances of x, say y \& z know each other. Then \{x, y, z\} form a club.
		\item No two acquaintances of x know each other. Then they form a group of strangers.
	\end{enumerate}
	\item x has at most 2 acquaintances there one ate least 3 people
\end{enumerate}

\section{Social Choice Function}

\subsection{Definition}
3 candidates A, B \& C:
\begin{itemize}
	\item 49\% of electorate A > B > C
	\item 48 \% of electorate B > A > C
	\item 3\% of electorate C > B > A
\end{itemize}

Given a collection of voters $v_1, v_2, ..., v_n$ and several candidates A, B, C, D, ...

Each voter ranks the candidates according to his preferences:

$A >^{v_2} B >^{v_2} C >^{v_2} D \qquad >^{v_i} is the ordering produced by the i^{th} voter$

\paragraph{Permutation}

(A, D, B, C) of the set of candidates \{A, B, C, D \}

Social choice function takes as an input voter's ordering and produces a consensus ordering and produces a consensus ordering $f(>^{v_1},>^{v_2},,...,>^{v_n}) = \quad >$

What conditions should a good SCF satisfy?
\begin{enumerate}
	\item \textbf{Unanimity}: If every voter prefers $\alpha$ to $\beta$ then the consensus ordering must rank $\alpha$ above $\beta$ \\
	$(\forall \vee (\alpha >^{v} \beta)) \to (\alpha > \beta)$
	\item \textbf{Independence on irrelevant alternatives (IAA)}
	The final relative ordering of $\alpha$ and $\beta$ should depend only on relative orderings of $\alpha$ and $\beta$ (If a candidate withdraws from election this doesn't affect the order of others). 
\end{enumerate}

Which social choice functions satisfy these properties?

What happens with majority? $\alpha > \beta$ if more than half of the voters prefer $\alpha$ to $\beta$:
\begin{itemize}
	\item $v_1 : A >^{v_1} B >^{v_1} C $
	\item $v_2 : C >^{v_2} A >^{v_2} B $
	\item $v_3 : B >^{v_3} C >^{v_3} A $
\end{itemize}

How does this work? There is a conflict here...

Dictatorship: For some fixed voter d we have ($\alpha > \beta$) if and only if ($\alpha >^d \beta$)

\subsection{Arrow's impossibility Theorem (1951)} 
The only social choice function satisfying unanimity and IIA is a dictatorship.

\paragraph{Proof}

Unanimity $\wedge$ IIA $\to$ dictatorship

Let $>$ satisfy these two properties $\beta$ is called a polarizing candidate if every voter ranks him.her at the very top or the very bottom of the list.

\textbf{Claim} A polarizing candidate ranks first or last in the consensus ordering $>$

\textbf{Proof} Suppose not $\alpha > \beta > \gamma $ where $\beta$ is a polarizing candidate

\begin{tabular}{| l | l | l | l | }
  $\beta$ & $\beta$ & $\alpha$  & $\gamma$\\
  $\alpha$ & $\gamma$ & $\gamma$  & $\alpha$\\
  $\gamma$ & $\alpha$ & $\beta$  & $\beta$\\
\end{tabular}

Switch $\alpha$ and $\gamma$ in voter's preferences so that every voter prefers $\gamma$ to $\alpha$. We should still have $\alpha > \beta > \gamma$ because relative positions of $\alpha$ and $\beta$ and relative positions of $\beta$ and $\gamma$ are unchanged. By unanimity we should now have $\gamma > \alpha$ (contradiction QED)


Choose a candidate $\beta$

\begin{tabular}{| l | l | l | l | }
  $\beta$ & $\beta$ & $\alpha$  & $\gamma$\\
  $\alpha$ & $\gamma$ & $\gamma$  & $\alpha$\\
  $\beta$ & $\beta$ & $...$  & $\beta$\\
  $v_1$ & $v_2$ & $...$  & $v_n$\\
\end{tabular}
$\to$
\begin{tabular}{| l | l | l | l | }
  $\beta$ & $-$ & $-$  & $-$\\
  $\alpha$ & $\gamma$ & $\gamma$  & $\alpha$\\
  $-$ & $\beta$ & $...$  & $\beta$\\
  $v_1$ & $v_2$ & $...$  & $v_n$\\
\end{tabular}

%%% missed

So there exists a voter $v^*$ so that
\begin{tabular}{| l | l | l | l | }
  $\beta$ & $-$ & $-$  & $-$\\
  $\alpha$ & $\gamma$ & $\gamma$  & $\alpha$\\
  $-$ & $\beta$ & $...$  & $\beta$\\
  $v_1$ & $v_2$ & $...$  & $v_n$\\
\end{tabular}

Goddammit. Disregard this last section (the whole theorem). I will fix it later.

\section{Proofs}

\subsection{The well-ordering principle}
\begin{itemize}
	\item \textbf{The well-ordering principle} \\
	Every non empty subset of non-negative integers has a smallest element.
	\item \textbf{The induction principle}\\
	 "P(n) is true for all natural numbers b"
\end{itemize}

\subsubsection{Proofs using the well-ordering principle}

\paragraph{Claim}
There exists subsets of non-negative rational numbers with no smallest element.

$\{x \in \mathbb{Q} | x > 1\}$ ($\mathbb{Q}$ is the set of rational numbers\}

Suppose $x_0<x_1$, $x_0 \in \mathbb{Q}$ is a smallest element of this set $x_0 = \frac{m}{n} \quad m > n$

(missed)
%$1 < x^1 = $
\paragraph{Proving the irrationality of $\sqrt{2}$}
\begin{itemize}
	\item \textbf{Theorem} $\sqrt{2}$ is irrational.
	\item \textbf{Proof} Suppose $\sqrt{2}$ is rational (Proof by contradiction) \\
		$c = \{m \in \mathbb{N} | \exists n \in \mathbb{N} (\sqrt{2} = m / n)\}$ \\
		Our assumption is equivalent to the statement $C \neq \emptyset$ \\
		By the well-ordering principle there exists $m_0$ the smallest element of C \\
		$\sqrt{2} = \frac{m_0}{n_0} \to 2 = \frac{m_0^2}{n_0^2} \to 2n_0^2 = m_0^2 \to m_0 = 2m' \to 2n_0^2 = 4m'^2 \to n_0^2 = 2m'^2 \to n_0 = 2n' \to (2n')^2  = 2m'^2 \to 2n'^2 = m'^2 \to \sqrt{2}n' = m' \to \sqrt{2} = \frac{m'}{n'} \to m \in C$ but $m' < m_0$
	\item	There is a contradiction as $m_0$ was chosen to be the smallest element of C.
\end{itemize}

\subsubsection{Method}
Structure of the proofs using well-ordering principle: \\
"P(n) is true for all positive integeres n" (In our theorem P(m) := "$\neg(\exists n \in \mathbb{N} \quad \sqrt{2} = \frac{m}{n})$"
	\begin{enumerate}
		\item $C = \{ n \in \mathbb{N} |$ P(n) is False $\}$
		\item Assume for a contradiction that $C \neq \emptyset$
		\item By the well-ordering principle we can choose $n_0$ the smallest element of C
		\item Obtain the contradiction to this choice (for example show that $n_0 \notin C$)
	\end{enumerate}

Theorem
Every positive integer bigger than 1 can be expressed as a product of prime numbers (being prime counts).

Statement
P(n) = "If n > 1, then n can be expressed as product of prime numbers

C = $\{n \in \mathbb{N} : n > 1$ n can be expressed as such a product\}

Choose $n_0$ to be a smallest element of C (We are using proof by contradiction)

\begin{itemize}
	\item Case 1: $n_0$ is prime \\
	This can't happen. $n_0$ by itself would be a valid product
	\item Case 2: $n_0$ is not prime \\
	      $n_0 = ab \qquad 1 < a,b < n_0 \qquad a,b \in \mathbb{N}$ \\
	      Therefore as $a, b \notin C$, so $a = p_1p_2p_3...p_k \quad p_i$ is prime and 
	      						$b = q_1q_2 ... q_i \quad q_j$ is prime \\
	       $n = p_1p_2p_3...p_k . q_1q_2q_3...q_k \quad$ So n is a product of primes \\
	       $n \notin C \to$ contradiction
\end{itemize}

\paragraph{What is the sum of the first n odd (+) integers?}

1 + 3+ 5 +... + (2n-1)

\begin{align*}
1 &= 1
1 + 3 &= 4
1 + 3 + 5 &= 9
1+ 3 + 5 + 7 &= 16
\end{align*}
It looks like the answer is $n^2$ (but this is not enough to prove it)

\paragraph{Theorem:} $1 + 3 + 5 + 7 + ... + (2n-1) = n^2$

\textbf{Proof}: Suppose for a contradiction that $n_0$ is the smallest positive integer for which this formula is false

$1 + 3 + 5 + ... + 2n -1 \neq n_0$

The formula is true for $n_0 -1 \qquad (n_0 \neq 1)$

$1 + 3 + 5 + ... + 2(n_0 -1) -1 = (n_0 - 1)^2$

So $1 + 3 + 5  + ...+ (2n-1) = (n_0 -1)^2 + 2n_0 -1 = n_0^2 \to$ contradiction!

\subsection{Induction}

\paragraph{Method}
"P(n) is true for all n $\in \mathbb{N}$ if
\begin{enumerate}
	\item Base of induction: "P(1) is true"
	\item Induction steps: "P(n-1) implies P(n) for all $n \geq 2$ \\
	(Equivalently  "P(n) implies P(n+1) for all $n \geq 1$"
\end{enumerate}

\subsubsection{A few examples}

\paragraph{Sum of n first Integers}
\begin{itemize}
	\item \textbf{Theorem}: $1 + 2 + ... +n = \frac{n(n+1)}{2}$
	\item \textbf{Proof}: By induction on n
	\item \textbf{Base case n =1}: $1 = \frac{1}{1+1}/2 = 1$
	\item \textbf{Induction step:} $P(n) \to P(n+1) for n \geq 1$\\
			$1 + 2 + ... + n = \frac{n(n+1)}{2}$
			$P(n+1): 1 + 2 + ... +n + (n+1) = \frac{n(n+1)}{2} + (n + 1) = (n + 1) \cdot ( \frac{n}{2} + 1) = \frac{(n+1) \cdot (n + 2)}{2} \Box$
\end{itemize}

\paragraph{$2^n \geq n^2$}
\begin{itemize}
	\item \textbf{Theorem}: $2^n \geq n^2 \forall n \geq 4$
	\item Base case: $n = 4, 2^4 = 16 = 4^2$
	\item Induction step: $(2^n \geq n^2) \to (2^{n+1} \geq (n+1)^2)$
		\begin{align*}
		2^{n+1} &= 2^n \cdot 2 \\
		&\geq 2n^2 = n^2 + n^2 \\
		&\geq n^2 + 4n \qquad (n \geq 4) \\
		&\geq n^2 + 2n + 1 = (n + 1)^2 \qquad (2n \geq 1) \\
		\end{align*}
\end{itemize}

\paragraph{A flawed induction proof}
\begin{itemize}
	\item Theorem: All horses are the same colour
	\item Base Case: One horse is the same colour as its self
	\item Induction step: Any n horse are the same colour. Any n + 1 horses are the same colour.
\end{itemize}

Flaw: P(n) does not imply P(n+1)
%%% Write this later

\section{Number Theory}

\paragraph{Definition}
Studies properties of integers: divisibility, primes. an integer a divides an integer b $if \exists x \in \mathbb{Z} (xa=b)$


Division with remainder: for any two integers $a > 0$ and b there exists integers q and d  so that $b = qa + r \to$ r is the remainder with $0 \leq r \leq a$

We write $a | b$, if and only if $r = 0$

A positive integer $p > 1$ is prime if the only positive integers dividing p are 1 and p

An integer $n > 1$ is composite if it is not prime

Expression of a positive as product of primes is called prime factorization. We have shown that every integer greater than 1 admits a prime factorization.

\paragraph{The fundamental theorem of arithmetic} Every integer greater than 1 admits unique prime factorization

\textbf{Proof using contradiction}: Suppose some n admits at least two distinct prime factorizations. Choose the minimum such n.

$n = p_1p_2p_3...p_k = q_q1_2...q_l$

We may assume that $p_1$ is the smallest prime among all the primes $p_i$ and $q_j$

Suppose $p_1 = q_1$ then $\frac{n}{p_i} = p_2...p_i = q_2 ... q_l$
It is a smaller number admitting two different factorizations

So $p_1 < q_1 so q1 = xp_1 + r \qquad 0 \leq r \leq p_1$

$pn = (xp_1 + r )q_2 ... q_i also we may assume q_2, q_3 , ... q_l \neq p_1$

$p_1 | n$ 

$n = xp_1p_2...q_l + rq_2 .... q_l$

$n is divisible by p_1$

$n > m > 1 m = rq_2...q_l$ is divisible by p

$m < n$ as $x > 0$ because $q_1 > p_1 > r$

We will sow that m also has two different prime factorizations 

$m = p_1m = p_1r_1r_2...r_s \to$ there exists a prime factorization of m which includes 

A contradiction to the choice of..

% missed


%%% Friday September 30th 2011

\subsection{Primes}

\paragraph{Fundamental theorem of arithmetic} Every integer greater than 1 can be uniquely expressed as a product of primes

$a=p_1^{r1}p_2^{r_2}...p_k^{r_k}$
$b = p_1^{s_1}p_2^{s_2}...p_k^{s_k}$

$1200 = 2^4\cdot 3 \cdot 5^2$

$[ab = p_1^{r_1 + s_1} p_2^{r_2 + s_2} ... p_k^{r_k + s_k}]$

$a | b$ if and only if $r_i \leq s_i$ for all $1 \leq i \leq k$

\begin{itemize}
    \item \textbf{Theorem} $\sqrt{2}$ is irrational
    \item \textbf{Proof}: Suppose $\sqrt{2}$ is not \\
    then $\sqrt{2} = \frac{m}{n} = 2 = \frac{m^2}{n^2}$ \\
    ($m = 2^r p_1^{r_1} \ldots p_i^{r_k}$)\\
    ($m = 2^s p_1^{s_1} \ldots p_i^{s_k}$)\\
    $(2n^2 = m^2$)\\
    ($2n^2 = 2^{2s+1} p_1^{2s_1} \ldots p_k^{2s_k}$)\\
    ($2m^2 = 2^{2r} p_1^{2r_1} \ldots p_k^{2r_k}$)\\
    Contradiction as ($2s+1 \neq 2r$)
    \item \textbf{Theorem:} $\sqrt{n}$ is rational for any integer n if and only if $n = k^2$ for some k
\end{itemize}

Proof: Exercise. Modify the proof for $\sqrt{2}$

\begin{itemize}
    \item \textbf{Theorem}: If a prime $p | ab$ then either p | a or p| b
    \item \textbf{Proof}: if p is not present in the prime factorization of either a or b, then it is not present in the prime factorization of $a \cdot b \cdot m$ and so $p \nmid a b \Box$
    \item Is this true when p is not prime? Suppose we have $p_1 \cdot p_2$ instead of $p_1$ then $p_1 | a$ or $p_1 | b \qquad$ $p_2 | 1$ or $p_2 | b$
\end{itemize}

--------------

$2,3,5,7,11,13,17,19,23,29,31,37,41,43$
\begin{enumerate}
    \item There are infinitely many prime numbers.
    \item Prime numbers are not everywhere dense in natural numbers. There are large gaps
\end{enumerate}

\begin{itemize}
    \item \textbf{Theorem} For any positive integer k there exists two consecutive prime numbers with difference $\geq k$
    \item \textbf{Proof} It suffices to exhibit a sequence of $\geq k$ consecutive composit numbers\\
            $n! = 1 \cdot 2 \cdot 3 \cdot 4 \cdot ... \cdot (n-1) \cdot n$\\
            Let $n = k + 1$\\
            $n! + 2, n! + 3, n! + 4, n! + 5,..., n! + n \to k$ consecutive integers all composite
            $n! + m$ is composite for all $2 \leq m \leq n \qquad m | n! + m \leftarrow m| n! \qquad 2 \leq m \leq n! + m$
    \item \textbf{Twin prime conjecture} There exist infinitely many pairs $p, p+2$ so that they are both primes. (This question has been asked more than 2000 years ago.)
    \item \textbf{The prime number theorem:} For a number n let $\pi(n)$ denote the number of primes $\leq n$. Then \\
            $\pi(n) ~ \frac{n}{\ln n} \qquad lim_{n \to \infty} \frac{\pi(n)}{n / \ln n} = 1$ \\
            Average gaps between primes are $\tilde \ln n$ 
    \item \textbf{Conjecture} For every positive integer n there exists a prime p $n^2 \leq p \leq (n+1)^2$
    \begin{enumerate}
        \item It is possible to efficiently test whether a number is prime or composite.
        \item It is believed not to be possible to efficiently produce prime factorisations.
    \end{enumerate}
\end{itemize}

\subsection{Greatest common divisors and linear combinations}

\paragraph{Die Hard 3 Problem} A jug containing precisely 4 gallons of water deactivates the bomb. He has a 3 gallon, a 5 gallon jug and 5 minutes. Given an a-gallon jug and a b-gallon jug, what ammounts can we get?

$a \leq b$ let (x,y) record current amounts of water in jugs of size a and b respectively

Here is an example of our approach
\begin{enumerate}
    \item (a,0) - fill in the first jug
    \item (0,a) - pour first jug into the second
    \item (a,a) - fill in the first one again
    \item (2a-b, b) - pour first into second
    \item (2a-b, 0) - empty the second jug
    \item (0, 2a-b)
    \item (a, 2a-b)
    \item (0, 3a-b) $\to$ John Mclane survives
\end{enumerate}

\subsection{Linear Combinations}
\begin{itemize}
    \item A linear combination of a and b is an integer expressible as $sa + tb$ where both s and t are integers.
    \item \textbf{Claim 1} The amounts of water in jugs are always linear combinations of a and b.\\
    (By induction on the number of operations performed)\\
    \item \textbf{Question:} Which numbers can we express as linear combinations of a and b?
\end{itemize}


%%%% Monday  October 3rd


\paragraph{Theorem} The amount of water in jugs is always a linear combination of a and b.

Let $L = \{m : m = sa+tb$ for some $s, t \in \mathbb{Z}$

\begin{enumerate}
    \item $0, a, b \in L$
		$0 = o \cdot a + 0 \cdot b$
		$a = 1 \cdot a + 0 \cdot b$
	\item $j_1, j_2 \in L$ then $j_1 + j_2 \in L, -j \in L$\\
\end{enumerate}

\paragraph{Proof}
By induction on \# steps performed
\begin{itemize}
	\item Base case (0 steps): $(0,0) \qquad 0 \in L$
	\item Induction step: Assume that after n steps we have amounts $j_1, j_2$ and $j_1, j_2 \in L$. we want to show that after the next step the amounts are still a linear combination.\\
	$(j_1, j_2) \to (0, j_2)$ or $(j_1, 0)$ or $(a, j_2)$ or $(j_1, b)$ or $(0, j_1 + j_2)$ or $(j_1 + j_2, 0)$ or $(j_1 + j_2 - b, b)$ or $(a, j_1 + j_2 -a)$\\
	$j_1 + j_2 - b \in L$ and $j_1 + j_2 - a$
	\item Theorem If $a \leq b$, $c \leq b$. Then if c is a linear combination of a and b, it is possible to measure exactly c liters.
	\item Proof: $c = sa + tb$ for some $s, t \in \mathbb{Z}$
	\begin{itemize}
		\item Case 1: $c = b$
		\item Case 2: $c < b$ We may assume that $s > 0, t \leq 0$\\
			  $c = (s + kb)a + (t - ka)b = sa + tb + kba - kab$\\
              Choose k large so that $s + kb > 0$\\
              $c = sa -tb \to$ fill in the jug with capacity a s times repeatedly and pour it into a jug with capacity b as soon as the jug with capacity b becomes full pour it out\\
              $sa = t'b + c'$\\
              $sa = tb + c$\\
              $0 \leq c, c' <b$\\
              $c' = sa - t'b \to$ amount we poured out \\
              $s \to$ total amount we took\\
              In the end we have amounts $(0, c')$\\
              \textbf{Example:}\\
              $b = 5, a = 3, c = 4$\\
              $4 = 3 \times 3 - 5$\\
	          $(0,0) \to (3,0) \to (0,3) \to (3,3) \to (1,5) \to (1,0) \to (0,1) \to (3,1) \to (0,4)$\\
              There are many more ways to achieve the same result.
        \end{itemize}
\end{itemize}
\paragraph{What about $b=6, a=3, c=4$} 
Do there exist integers s and t such as 3s + 6t = 4?
d is a common divisor of a and b if d | a and d|b\\
\paragraph{Definition}
    If d is a \textbf{common divisor} of a \& b then every linear combination of a \& b is divisible by d.

The largest common divisor of a and b is denoted by gcd(a, b) and is called the \textbf{greatest common divisor}.

\paragraph{Theorem}
    $gcd(a, b)$ is the smallest positive linear combination of a and b. 
\paragraph{Proof}
    $d = gcd(a,b)$, let m be the smallest positive linear combination of an and b \\
    $m = sa + tb$. We want to show that d = m.
    \begin{enumerate}
        \item $d \leq m$\\
            $d | m$ Because d is a common divisor of a and b and m is a linear combination\\
            $d \leq m$ as they are both positive
        \item $m \leq d$\\
            It is enough to show that m is a common divisor of a and b.\\
            We'll show $m | a$ (showing $m | b$ is exactly the same).\\
            \textbf{Proof}: 
            \begin{itemize}
                \item Suppose not ${m \nmid a}$ dividing with remainder \\
                    $a = qm + r \qquad 0 < r <m$ ($r \neq 0$ because $m \nmid a$) % btw I have no idea how to make the not divide bar)
                \item $r = a - qsa - qtb = a(1-qs) + (-qt)b$ \\
                \item r is a linear combination of a \& b, contradicting the choice of m.   
            \end{itemize}
\end{enumerate}

\paragraph{Corollary}
    An integer c is a linear combination of a and b if and only if gcd(a,b) | c \\
    \textbf{Proof}:
    \begin{itemize}
        \item If $c = sa + tb$ then $gcd(a,b) | sa$ and $gcd(a,b) | tb$ so $gcd(a,b) | c$
        \item On the other hand, we know $gcd(a, b) = s'a + t'b$ for some $s', t' \in \mathbb{Z}$
        \item If $c = d gcd(a, b) then c = d(s'a + t'b) = (ds')a + (dt')b$
    \end{itemize}



%%% wednesday october 5th

\paragraph{Midterm information}
    \begin{itemize}
        \item \textbf{Material}: 
        \begin{enumerate}
            \item Logic and Proofs
            \item Number theory (up to modular arithmetic at the end of the week)
        \end{enumerate}
    \end{itemize}

\subsection{Greatest common divisors}
Let a and b be positive integers.
$c = gcd(a,b)$ is the largest integer c such as $c | a$ \& $c | b$.

\begin{itemize}
    \item \textbf{Theorem}:gcd(a, b) is the smallest positive linear combination of a \& b.
    \item \textbf{Corollary}: c is a linear combination of a \& b if and only if $gcd(a, b) | c$
    \item \textbf{Theorem}:
        \begin{enumerate}
            \item $gcd(a, b)$ is divisible by every common divisor of a \& b
            \item $gcd(ka, kb) = k \cdot gcd(a, b)$ for any positive integer k
            \item $gcd(a, b) = 1, \quad gcd(a,c) = 1 \to gcd(a,bc) = 1$\\
                    If a and b do not have a common divisor and a and c do not have a common divisor then a and bc do not have anything in common either.
            \item $gcd(a, b) = 1, a | bc \to a | c$
            \item $a = qb + r \to gcd(a, b) = gcd(b, r)$
        \end{enumerate}
\end{itemize}

\paragraph{Proof of 3}

$s_1 a + t_1b = 1 \qquad s_2a + t_2c = 1$

It suffices to show that 1 is a linear combination of a and bc.

$1 = (s_1 \cdot a + t_1 \cdot b)(s_2 \cdot a + t_2 \cdot c) = a(s_1 \cdot s_2 \cdot a + s_1 \cdot t_2 \cdot c + s_2 \cdot t_1 \cdot b) + bc( t_1 \cdot t_2 \qquad)$ (can also be derived from prime decomposition)

\paragraph{Proof of 5}

$d_1 = gcd(a,b) \qquad d_2 = gcd(b,r)$

\begin{itemize}
    \item $d_1 \leq d_2$ It is enough to show that $d_1 | b$ and $d_1 | r$\\
        $d_1 | b$ is trivial since $d_1$ is gcd(a, b) \\
        $r = a - qb$ a is divisble by $d_1$ and b is divisible by $d_1$ therefore $d_1$ divides r
    \item $d_2 \leq d_1$ It is enough to show that $d_2 | a$ and $d_2 | b$\\
        $d_2| b$ is trivial \\
        $a = r + qb$ since $d_2$ is a divisor of r and b then $d_2$ is a linear combination of r and b and $d_2 | a$ 
\end{itemize}

\section{Euclid's algorithm}

\subsection{Computing gcd with prime factorization}
    \begin{itemize}
        \item $a = p_1^{r_1} \cdot p_2^{r_2} ... p_k^{r_k}$
        \item $b = p_1^{s_1} \cdot p_2^{s_2} ... p_k^{s_k} = p_1^{r_1} \cdot p_2^{r_2} ... p_k^{s_k}$ 
        \item $gcd(a, b) = p_1^{min(r_1, s_1)} \cdot p_2^{min(r_2, s_2)} ... p_k^{min(r_k, s_k)}$

        \item \textbf{Example}: $1200 = 2^4 \cdot 3 \cdot 5^2$
        \item $280= 2^3 \cdot 5 \cdot 7$ $a = p_1^{r_1} \cdot p_2^{r_2} ... p_k^{s_k}$=
        \item $gcd(1200, 280) = 2^3 \cdot 5 = 40$
    \end{itemize}

\subsection{Computing gcd with Euclid's algorithm}
$a = qb + r \qquad gcd(a,b) = gcd(b, r)\\
gcd(962, 230) = \qquad 962 = 4 \cdot 230 + 42\\
gcd(230, 42) = \qquad 230 = 5 \cdot 42 + 20\\
gcd(42, 20) = \qquad 42 = 20 \cdot + 2\\
gcd(20, 2) =\\
2
$

\subsection{Statement of Euclid's algorithm}
GCD(a, b) \\
\textbf{Input}: integers a \& b (in binary)\\
\textbf{Steps}:\\
\begin{enumerate}
    \item $a \geq b$
    \item Divide with remainder $a = qb + r, 0 \leq r < b$
    \item $If r = 0 \to \textbf{output}: b$
    \item Otherwise, run GCD(b,r)
\end{enumerate}

\subsection{Analysis of Euclid's algorithm}
    \begin{enumerate}
        \item It is valid by part 5 of the preceding theorem ($a = qb + r \to gcd(a, b) = gcd(b, r)$)
        \item It terminates in at most a + b  $\to$ in each recursive step we replace a by r.\\
            So the sum of the inputs decreases.
        \item Is it efficient (polytime)?\\
            We want to show that it terminates in $O((log a + log b)^k)$
        \begin{itemize}
            \item \textbf{Claim}: $a = qb + r \qquad 0 \leq r \leq b, a \geq b$ then $ab \geq 2br$
            \item \textbf{Proof}: We need to show that $a\geq 2r$ \\
                $q \geq 1 \to a \geq b+r \to a \geq r+r = 2r$
            \item The claim implies that the product of the inputs is reduced by at least a factor of 2 in each step. \\
                So there are at most $\log(ab)$ steps in recursion \\
                $\log (ab) = \log a + \log b \to$ \textbf{linear algorithm}
        \end{itemize}
    \end{enumerate}

\subsection{Expressing gcd(a,b) as a linear combination of a \& b}
$gcd(962, 230) = \qquad 962 = 4 \cdot 230 + 42\\
gcd(230, 42) = \qquad 230 = 5 \cdot 42 + 20\\
gcd(42, 20) = \qquad 42 = 20 \cdot + 2\\
gcd(20, 2) =\\
2
$

\begin{align*}
2 &= 42 - 2 \cdot 20 \\
&= 42 - 2 \cdot (230 - 5 \cdot 42)\\
&= 11 \cdot 42 - 2 \cdot 230\\
&= 11 \cdot (962 - 4 \cdot 230) - 2 \cdot 230\\
&= 11 \cdot 962 - 46 \cdot 230
\end{align*}

%%% Friday October 7th

\begin{verbatim}
Wesley's notes. Will format later.

== Euclid's Algorithm ==

The algorithm takes at most <> iterations to terminate.

Each individual step can also be performed quickly. (Division with remainder)

Arithmetic operations: additions, multiplication, division with remainder take time polynomial in input size.

Input is usually in binary.

=== Adding ===

<>

<>

<>

Adding a & b takes <>

=== Multiplication ===

Multiplication is similar. At most <>

=== Division ===

Division with remainder can also be done efficiently

Each individual step can also be performed quickly. (Division with remainder)

Aritmetic operations addition, multiplication, division with remainder take time polynomial in input size.

\end{verbatim}


\subsection{Homework problem}:

Show that deciding whether $ax^2 + by = c$ has an integer solution for given $a, b \leq c$ in NP.

Size of the input; $log_2{a} + log_2{b} + log_2{c}$

Certificate $x \& y$ such that $ax^2 + by = c$

\paragraph{Essence of the problem}: If there exists x \& y which solve the solution, then there exist $x_0, y_0$ so that $ax_0 + b \cdot y \cdot b_{y_0}  = c$ and $x_0$ and $y_0$ are not too large 

$x_0y_0 = 0((a+b+c)^n)$

\paragraph{General diophantine equation}:

$P(x_1, x_2, ..., x_k) = 0 \qquad$ where P is polynomial with integer coefficients.

E.g. $x_1x_2x_3 - 5x_1 + 1000 = 0$

Not in NP in general, there is no systemic algorithm to figure out if it has a solution or not.

Euclid's algorithm takes at most $log_2{a} + log_2{b}$ steps (but maybe it always terminates in 5 or $\log(\log a)$

In worst case scenario is takes $\Omega(log_2{a} + log_2{b})$ steps

\textbf{Example}: Fibonacci numbers: $F_1 = 1, F_2 = 1, F_3 = 2, F_4 = 3$

1, 1, 2, 3, 5, 8, 13, 21 $ \to F_n = F_{n-1} + F_{n-2}$

Running Euclid's algorithm to compute $gcd(F_n, F_n+1)$

$F_n = F_{n-1}+ F_{n-2}$

$F_{n-1} = F_{n_2} + F_{n-3}$

$F_n = {{1 + \sqrt{5}}/2^{n+1}}/\sqrt{5}$

%1 + \sqrt{5}%/2 = 1.6....$

\section{Modular arithmetic}

\subsection{Notation}

We say that $a$ is \textbf{congruent} to $b$ \textbf{modulo} $m$ if $m | a - b$. We note it $a \equiv b (mod m)$

$rem(a, m)$: the remainder of $a$ after division by $m$

$a = km + rem(a, m)$

$0 \leq rem(a, m) < m$

\textbf{Fact}: $a \equiv b (mod m)$ if and only if $rem(a, m) = rem(b, m)$
\textbf{Proof}: 
$a = k_1m + rem(a, m)$

$b = k_2m + rem(b, m)$

$0 \leq rem(a, m), rem(b, m) < m$

If $rem(a, m) = rem(b, m)$ then $a - b = (k_1 - k_2)m$. Therefore $a =b(mod m)$

$a - b = (k_1 - k_2)m + (rem(a, m ) - rem(b, m)$

Therefore r$em(a, m) = rem(b, m) = 0$.

In many senses you can operate with congruences as with equations.

\textbf{Theorem}: (Properties of congruences)
\begin{enumerate}
    \item \textbf{Reflexitivity} $a \equiv a (\mod m)$
    \item \textbf{Symmetry} $a \equiv b (\mod m)$ if and only if $b \equiv a (\mod m)$
    \item \textbf{Transitivity} if $a \equiv b (\mod m) \quad \& \quad b \equiv c (\mod m)$ then $a \equiv c (\mod m)$
\end{enumerate}

Suppose $a \equiv b (\mod m), c \equiv d (\mod m)$:

Then:
\begin{enumerate}
    \item $a + b \equiv b + d (\mod m)$
    \item $ac \equiv bd (\mod m)$
\end{enumerate}

\textbf{Proof}: 
1, 2, 3: is based on the preceding fact
3. If $rem(a, m) = rem(b, m) and rem(b, m) = rem(c, m), then rem(a, m) = rem(c, m).
4. $m | {a - b}$, $m | c - d$ therefore $m | (a-b) + (c-d) = (a + c) - (b + d)
















\end{document}
